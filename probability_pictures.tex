\documentclass[12pt]{article}

%\usepackage{tikz} % картинки в tikz
\usepackage{microtype} % свешивание пунктуации

% \usepackage{listings}
\usepackage{minted,tikz}

\usepackage{array} % для столбцов фиксированной ширины

\usepackage{indentfirst} % отступ в первом параграфе

\usepackage{sectsty} % для центрирования названий частей
\allsectionsfont{\centering}

\usepackage{amsmath, amssymb, amsthm} % куча стандартных математических плюшек

\usepackage{amsfonts}

\usepackage{comment}

\usepackage[top=2cm, left=1.2cm, right=1.2cm, bottom=2cm]{geometry} % размер текста на странице

\usepackage{lastpage} % чтобы узнать номер последней страницы

\usepackage{enumitem} % дополнительные плюшки для списков
%  например \begin{enumerate}[resume] позволяет продолжить нумерацию в новом списке
\usepackage{caption}

\usepackage{physics}

\usepackage{hyperref} % гиперссылки

\usepackage{multicol} % текст в несколько столбцов


\usepackage{fancyhdr} % весёлые колонтитулы
\pagestyle{fancy}
\lhead{Картинки по теории вероятностей и статистике}
\chead{}
\rhead{}
\lfoot{}
\cfoot{}
\rfoot{}
\renewcommand{\headrulewidth}{0.4pt}
\renewcommand{\footrulewidth}{0.4pt}

\let\P\relax
\DeclareMathOperator{\P}{\mathbb{P}}
\DeclareMathOperator{\Cov}{\mathbb{C}ov}
\DeclareMathOperator{\E}{\mathbb{E}}
\DeclareMathOperator{\Var}{\mathbb{V}ar}
\newcommand{\cN}{\mathcal{N}}

\usepackage{todonotes} % для вставки в документ заметок о том, что осталось сделать
% \todo{Здесь надо коэффициенты исправить}
% \missingfigure{Здесь будет Последний день Помпеи}
% \listoftodos - печатает все поставленные \todo'шки


% более красивые таблицы
\usepackage{booktabs}
% заповеди из докупентации:
% 1. Не используйте вертикальные линни
% 2. Не используйте двойные линии
% 3. Единицы измерения - в шапку таблицы
% 4. Не сокращайте .1 вместо 0.1
% 5. Повторяющееся значение повторяйте, а не говорите "то же"

\usepackage{tikz}
\usetikzlibrary{automata, arrows, positioning, calc}


\usepackage{fontspec}
\usepackage{polyglossia}

\setmainlanguage{english}
\setotherlanguages{russian}

% download "Linux Libertine" fonts:
% http://www.linuxlibertine.org/index.php?id=91&L=1
\setmainfont{Linux Libertine O} % or Helvetica, Arial, Cambria
% why do we need \newfontfamily:
% http://tex.stackexchange.com/questions/91507/
\newfontfamily{\cyrillicfonttt}{Linux Libertine O}

\AddEnumerateCounter{\asbuk}{\russian@alph}{щ} % для списков с русскими буквами
% \setlist[enumerate, 2]{label=\asbuk*),ref=\asbuk*}




\begin{document}


\begin{center}
\begin{tikzpicture}[->, >=stealth', auto, semithick, node distance=3cm]
\tikzstyle{every state}=[fill=white,draw=black,thick,text=black,scale=1]
\node[state]    (A)               {$S$};
\node[state]    (B)[right of=A]   {$6$};
\node[state]    (C)[right of=B]   {$62$};
\node[state]    (D)[right of=C]   {$626$};
\path
(A) edge[loop below]    node{}           (A)
    edge                node{}           (B)
(B) edge                node{}           (C)
    edge[loop below]    node{}           (B)
    edge[bend right]    node{}           (A)
(C) edge                node{}           (D)
    edge[bend right]    node{}           (B)
    edge[bend right=40] node{}           (A);
\end{tikzpicture}
\end{center}

\begin{minted}[linenos]{latex}
\begin{tikzpicture}[->, >=stealth', auto, semithick, node distance=3cm]
\tikzstyle{every state}=[fill=white,draw=black,thick,text=black,scale=1]
\node[state]    (A)               {$S$};
\node[state]    (B)[right of=A]   {$6$};
\node[state]    (C)[right of=B]   {$62$};
\node[state]    (D)[right of=C]   {$626$};
\path
(A) edge[loop below]    node{}           (A)
    edge                node{}           (B)
(B) edge                node{}           (C)
    edge[loop below]    node{}           (B)
    edge[bend right]    node{}           (A)
(C) edge                node{}           (D)
    edge[bend right]    node{}           (B)
    edge[bend right=40] node{}           (A);
\end{tikzpicture}
\end{minted}



\end{document}
